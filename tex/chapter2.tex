% !TeX root=../main.tex
\chapter{مروری بر منابع}
%\thispagestyle{empty} 


درمان با استفاده از نور ریشه در دوران باستان دارد. در مصر، یونان، و هند باستان، از نور خورشید برای درمان بیماری‌هایی مانند ویتیلیگو، پسوریازیس و حتی اختلالات روانی استفاده می‌شد. این روش که به "هلیوتراپی"\LTRfootnote{heliotherapy}
 به معنی نوردرمانی معروف بود، اغلب با استفاده از گیاهان دارویی تقویت می‌شد. برای مثال، مصریان باستان پودر گیاهانی همچون آمی ماجوس\LTRfootnote{Ammi majus}
 را روی پوست بیماران می‌گذاشتند و آن‌ها را در معرض آفتاب قرار می‌دادند تا بهبودی حاصل شود. در هند نیز، گیاهان دارویی خاص مانند پسورالیا کوریلیفولیا\LTRfootnote{Psoralea corylifolai}
 برای درمان ویتیلیگو\LTRfootnote{vitiligo}
تجویز می‌شد. با افول امپراتوری‌های باستانی و ظهور قرون وسطی، استفاده از نوردرمانی کمرنگ شد\cite{pdt-Lu2009581,pdt-theory-to-application}.

در دوران مدرن، استفاده از نور در درمان مجدداً با مطالعات علمی احیا شد. در اواخر قرن نوزدهم، آرنولد ریکلی\LTRfootnote{Arnold Rikli}
، طبیب سوئیسی، مفهوم نوردرمانی را با ترکیب روش‌های طبیعی بازتعریف کرد. او توجه جهانی را به قدرت شفابخشی نور معطوف ساخت. نیلز فینسن\LTRfootnote{ Niels Ryberg Finsen}، پزشک دانمارکی، گام مهمی در این مسیر برداشت و با استفاده از نور مصنوعی برای درمان بیماری‌های پوستی مانند سل پوستی، جایزه نوبل پزشکی را به دست آورد. او نشان داد که اشعه فرابنفش می‌تواند در بهبود بیماری‌های پوستی نقش مؤثری داشته باشد.\cite{pdt-theory-to-application}.

درمان فتودینامیک
(\lr{PDT}\LTRfootnote{Photodynamic therapy})
با اکتشافات اولیه‌ای که در اواخر قرن نوزدهم میلادی انجام شد، وارد دوران مدرن شد. یکی از مهم‌ترین این اکتشافات توسط اسکار راب\LTRfootnote{Oscar Raab}، دانشجوی پروفسور هاینریش تاپاینر\LTRfootnote{Hermann von Tappeiner} در دانشگاه لودویگ ماکسیمیلیان مونیخ، صورت گرفت. او مشاهده کرد که تأثیر سمی رنگ اکریدین بر پارامسی‌ها در روزی که طوفان رعد و برق رخ داده بود، به شکل چشمگیری کاهش یافت. راب نتیجه گرفت که نور به نحوی باعث فعال شدن رنگ اکریدین و افزایش اثر سمی آن بر پارامسی‌ها می‌شود. این یافته، اساس علمی درمان فتودینامیک را شکل داد. سپس، پروفسور تاپاینر که از پیشگامان زیست‌فوتوشیمی بود، در سال ۱۹۰۴ اصطلاح «عمل فتودینامیک»(\lr{Photodynamische Wirkung}) را معرفی کرد و گروه او تحقیقات گسترده‌ای را بر روی استفاده از رنگ‌های مختلف مانند ائوزین و فلورسئین برای درمان تومورها و بیماری‌های پوستی آغاز کردند. نتایج این تحقیقات، خصوصاً در درمان کارسینوم بازال سل و بیماری‌های پوستی مانند لوپوس، نشان‌دهنده موفقیت‌آمیز بودن این روش‌ها بود\cite{pdt-theory-to-application,pdt-Lu2009581}.

در دهه ۱۹۷۰، داگرتی\LTRfootnote{Dougherty} و همکارانش مشتقی از هماتوپورفیرین را توسعه دادند که در آزمایشگاه و مدل‌های حیوانی قادر به تخریب سلول‌های سرطانی بود. هماتوپورفیرین که در ابتدا توسط شِرِر\LTRfootnote{Scherer} در سال ۱۸۴۱ به عنوان محصولی ناخالص از خون خشک‌شده تولید شد، دارای خواص فتودینامیک قوی‌ای بود. مطالعات اولیه نشان داد که این ماده تحت تأثیر نور باعث آسیب بافتی می‌شود. اگرچه آزمایش انسانی اولیه توسط فردریش مایر بتز
\LTRfootnote{Friedrich Meyer-Betz}
در سال ۱۹۱۲ به دلیل اثرات جانبی شدید ناموفق بود؛ اما پژوهش‌ها در این حوزه ادامه یافت. در دهه ۱۹۵۰، دانشمندان مشاهده کردند که این ماده در تومورها تجمع می‌یابد و توانایی درمان سرطان را دارد\cite{pdt-theory-to-application}. 

این پیشرفت‌ها منجر به تأیید اولین روش رسمی درمان فتودینامیک توسط سازمان غذا و داروی آمریکا (\lr{FDA}) در سال ۱۹۹۵ برای درمان سرطان مری شد. امروزه با پیشرفت فناوری‌های تصویربرداری و استفاده از روش‌های هدایت‌شده مبتنی بر تصویر، دقت و کارایی
\lr{PDT}
به‌طور چشمگیری افزایش یافته و این روش به درمانی کم‌تهاجمی و مؤثر برای انواع سرطان تبدیل شده است\cite{pdt-theory-to-application,pdt-Lu2009581}.
 
 
 
 
 
 
 
 
 
 
 
 
 
 
 
 
 
 
 
 
 
 
 
 
 
 
 
 
 
 
 
 
 
 
 
 
 
 
 
 
 
 
 
 
 
 